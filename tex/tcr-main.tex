\documentclass[10pt,a4paper,ngerman]{article}

\usepackage[utf8]{inputenc}
\usepackage[T1]{fontenc}
\usepackage[safe,warn]{textcomp}
\usepackage{tgtermes}
\usepackage[scaled]{berasans}
\renewcommand*\ttdefault{txtt}

\usepackage{babel}

\usepackage{amsmath}
\usepackage{amsfonts}
\usepackage{amssymb}

\usepackage{listings}

\title{Team Contest Reference}
\author{Universität zu Lübeck}
\begin{document}
\lstset{basicstyle=\ttfamily\footnotesize,numbers=left,numberstyle=\tiny,tabsize=2,numbersep=5pt}
\maketitle

\section{Mathematische Algorithmen}
\subsection{Primzahlen}
\subsubsection{Sieb des Eratosthenes}
\lstinputlisting[language=Java]{eratosthenes.java}
\subsubsection{Primzahlentest}
\lstinputlisting[language=Java]{isprim.java}
\subsection{Binomial Koeffizient}
\lstinputlisting[language=Java]{binomial.java}

\section{Datenstukturen}
\subsection{Fenwick Tree (Binary Indexed Tree)}
\lstinputlisting[language=Java]{fenwick.java}

\section{Graphenalgorithmen}
\subsection{Topologische Sortierung}
\lstinputlisting[language=Java]{toposort.java}
\subsection{Prim (Minimum Spanning Tree)}
\lstinputlisting[language=c]{prim.c}

\section{Geometrische Algorithmen}
\subsection{Graham Scan (Convex Hull)}
\lstinputlisting[language=java]{graham.java}

\section{Verschiedenes}
\subsection{Potenzmenge}
\lstinputlisting[language=java]{powerset.java}
\end{document}
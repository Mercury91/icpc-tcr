\documentclass[10pt,a4paper,ngerman]{article}

\usepackage[utf8]{inputenc}
\usepackage[T1]{fontenc}
\usepackage[safe,warn]{textcomp}
\usepackage{tgtermes}
\usepackage[scaled]{berasans}
\renewcommand*\ttdefault{txtt}

\usepackage{babel}

\usepackage{amsmath}
\usepackage{amsfonts}
\usepackage{amssymb}

\usepackage{listings}

\title{Team Contest Reference}
\author{Universität zu Lübeck}
\begin{document}
\lstset{basicstyle=\ttfamily\footnotesize,numbers=left,numberstyle=\tiny,tabsize=2,numbersep=5pt}
\maketitle

\section{Mathematische Algorithmen}
\subsection{Primzahlen}
\subsubsection{Sieb des Eratosthenes}
\lstinputlisting[language=Java]{eratosthenes.java}
\subsubsection{Primzahlentest}
\lstinputlisting[language=Java]{isprim.java}
\subsection{Binomial Koeffizient}
\lstinputlisting[language=Java]{binomial.java}
\section{Mathematisch Formeln und Gesetze}
\subsection{Catalan}
$C_n = \frac1{n+1}\binom{2n}{n}=\prod_{k=2}^n (n+k)/k$\\
$C_{n+1} = \frac{4n+2}{n+2}C_n=\sum_{k=0}^{n}C_kC_{n-k}$
\subsection{kgV und ggT}
$ggT(n,m)\cdot kgV(m,n)=|m\cdot n|$
\subsection{Kreuzprodukt}
$\vec{a}\times\vec{b}
  =
  \begin{pmatrix}a_1 \\ a_2 \\ a_3\end{pmatrix}
  \times
  \begin{pmatrix}b_1 \\ b_2 \\ b_3 \end{pmatrix}
  =
  \begin{pmatrix}
    a_2b_3 - a_3b_2 \\
    a_3b_1 - a_1b_3 \\
    a_1b_2 - a_2b_1
  \end{pmatrix}$
\subsection{Orthogonale Projektion}
$r_0:$ Ortsvektor; $u:$ Richtungsvektor; $n:$ Normalenvektor\\
$P_g(\vec x) =  \vec r_0 + \frac{( \vec x - \vec r_0 ) \cdot \vec u}{\vec u \cdot \vec u} \, \vec u$\\
$P_g(\vec x) = \vec x - \frac{( \vec x - \vec r_0 ) \cdot \vec n}{\vec n \cdot \vec n} \, \vec n$(nur 2D bzw. 3D auf Ebene)\\
\subsection{Dreicksfläche}
$F=\sqrt{s(s-a)(s-b)(s-c)};\,s=\frac{a+b+c}{2}$
\section{Datenstukturen}
\subsection{Fenwick Tree (Binary Indexed Tree)}
\lstinputlisting[language=Java]{fenwick.java}

\section{Graphenalgorithmen}
\subsection{Topologische Sortierung}
\lstinputlisting[language=Java]{toposort.java}
\subsection{Prim (Minimum Spanning Tree)}
\lstinputlisting[language=c]{prim.c}

\section{Geometrische Algorithmen}
\subsection{Graham Scan (Convex Hull)}
\lstinputlisting[language=java]{graham.java}
\subsection{Punkt in Polygon}
\lstinputlisting[language=java,firstline=27,firstnumber=1,lastline=59]{PointInPoly.java}
\section{Verschiedenes}
\subsection{Potenzmenge}
\lstinputlisting[language=java]{powerset.java}
\subsection{LongestCommonSubsequence}
\lstinputlisting[language=c++]{longestCommonSubseq.cpp}
\subsection{LongestCommonSubstring}
\lstinputlisting[language=java,firstline=27,firstnumber=1,lastline=66]{LongestSubstring.java}
\subsection{Permutation \& Sequenzen}
\lstinputlisting[language=java]{PermsAndSequ.java}
\end{document}
